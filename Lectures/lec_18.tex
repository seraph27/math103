\lecture{18}{}{}

\begin{exercise}
    $G_1 = \mathbb{Z}_2 \times \mathbb{Z}_{12}$ and $G_2 = \mathbb{Z}_4 \times \mathbb{Z}_6$.
    Are these two groups isomorphic?
\end{exercise}
\begin{answer}
    We can use the theorem that $\mathbb{Z}_{mn} \cong \mathbb{Z}_m \times \mathbb{Z}_n$ if and only if $\gcd(m,n) = 1$.\\
    $G_1 = \mathbb{Z}_2 \times \mathbb{Z}_{43} \cong \mathbb{Z}_2 \times \mathbb{Z}_4 \times \mathbb{Z}_3$\\
    $G_2 = \mathbb{Z}_4 \times \mathbb{Z}_6 \cong \mathbb{Z}_2 \times \mathbb{Z}_4 \times \mathbb{Z}_3$\\
    $\Rightarrow G_1 \cong G_2$.
\end{answer}

\begin{theorem}[Finite Version]
    Let $G$ be a finite abelian group. Then $G$ is isomorphic to a direct product of cyclic groups of prime power order. 
    That is, 
    \[G \cong \mathbb{Z}_{p_1^{n_1}} \times \cdots \times \mathbb{Z}_{p_k^{n_k}}\]
    And, 
    \[|G| = p_1^{n_1} \cdots p_k^{n_k}\]
    where $p_1, \ldots, p_k$ are prime numbers and $n_1, \ldots, n_k$ are positive integers.\\
    Moreover, this decomposition is unique up to the order of the factors.
\end{theorem}

\begin{eg}
    Find all abelian groups up to isomorphosm of order 720. $720 = 2^4 \cdot 3^2 \cdot 5$.\\
\end{eg}
\begin{answer}
    By the theorem above, list out the primary factor representation of a group of order 720.\\
    \begin{tabular}{|c|c|c|}
        \hline
        $2^4$ & $3^2$ & $5$ \\ \hline
        $\mathbb{Z}_{16}$ & $\mathbb{Z}_9$ & $\mathbb{Z}_5$ \\ \hline
        $\mathbb{Z}_{8} \times \mathbb{Z}_2$ & $\mathbb{Z}_9$ & $\mathbb{Z}_5$ \\ \hline
        $\mathbb{Z}_{4} \times \mathbb{Z}_4$ & $\mathbb{Z}_9$ & $\mathbb{Z}_5$ \\ \hline
        $\mathbb{Z}_{4} \times \mathbb{Z}_2 \times \mathbb{Z}_2$ & $\mathbb{Z}_9$ & $\mathbb{Z}_5$ \\ \hline
        $\mathbb{Z}_{2} \times \mathbb{Z}_2 \times \mathbb{Z}_2 \times \mathbb{Z}_2$ & $\mathbb{Z}_9$ & $\mathbb{Z}_5$ \\ \hline
        $\mathbb{Z}_{16}$ & $\mathbb{Z}_3 \times \mathbb{Z}_3$ & $\mathbb{Z}_5$ \\ \hline
        $\mathbb{Z}_{8} \times \mathbb{Z}_2$ & $\mathbb{Z}_3 \times \mathbb{Z}_3$ & $\mathbb{Z}_5$ \\ \hline
        $\mathbb{Z}_{4} \times \mathbb{Z}_4$ & $\mathbb{Z}_3 \times \mathbb{Z}_3$ & $\mathbb{Z}_5$ \\ \hline
        $\mathbb{Z}_{4} \times \mathbb{Z}_2 \times \mathbb{Z}_2$ & $\mathbb{Z}_3 \times \mathbb{Z}_3$ & $\mathbb{Z}_5$ \\ \hline
        $\mathbb{Z}_{2} \times \mathbb{Z}_2 \times \mathbb{Z}_2 \times \mathbb{Z}_2$ & $\mathbb{Z}_3 \times \mathbb{Z}_3$ & $\mathbb{Z}_5$ \\ \hline
    \end{tabular}
\end{answer}

\begin{definition}
    We define torsion and torsion-free subgroups of a group as follows:\\
    \begin{enumerate}
        \item \textbf{Torsion Subgroup:} \\
        The \textit{torsion subgroup} of a group \( G \), denoted \( T(G) \), is defined as:
        \[
        T(G) = \{ g \in G \mid \text{there exists } n \in \mathbb{N} \text{ such that } g^n = e \}
        \]
        where \( e \) is the identity element in \( G \). It consists of all elements of \( G \) with finite order.

        \item \textbf{Torsion-Free Subgroup:} \\
        A \textit{torsion-free subgroup} of \( G \) contains only elements with infinite order, meaning:
        \[
        g^n \neq e \text{ for any nonzero integer } n
        \]
        except for \( g = e \) itself.
    \end{enumerate}
\end{definition}

\section{Cosets \& the Theorem of Lagrange}

\begin{remark}
    Let $G \cong \mathbb{Z}_n$ and $G = \langle a \rangle$ \\
    If $H$ is a subgroup of $G$, then $|H|$ divides $|G|=n$.\\
\end{remark}
\begin{proof}
    Let $H = \langle a^s \rangle$, then $(a^s)^{|H|} = e = a^n$.\\
    $(a^s)^{|H|} = \underbrace{a^sa^s \cdots a^s}_{|H|} = a^{s|H|} = a^n$\\
    So $s|H| = n$, and $|H|$ divides $n$.
\end{proof}

\begin{theorem}[Lagrange's Theorem]
    Let $G$ be a finite group and $H$ be a subgroup of $G$. Then $|H|$ divides $|G|$.\\
    Moreover, the number of left cosets of $H$ in $G$ is $\frac{|G|}{|H|}$.
\end{theorem}

\begin{corollary}
    Let $G$ be a group and $|G| = p = $prime. Then $G$ is cyclic $\cong \mathbb{Z}_p$.
\end{corollary}
\begin{proof}
    $|G| = p$ is prime. Let $H = \langle a \rangle$ be a subgroup of $G$.\\ 
    By Lagrange's Theorem, $|H|$ divides $|G| = p$.\\ So $|H| = 1$ or $p$.\\ So $H = G$.\\

    The proof is based on cosets which we will see later.\\ 
\end{proof}

\begin{definition}
    Let $G$ be a group and $H$ be a subgroup of $G$. We define a partition to have equivalence relation "$\sim$" on $G$ as follows:
    \[a \sim b \Leftrightarrow a^{-1}b \in H\]
    \begin{enumerate}
        \item Reflexive: $a \sim a$ since $a^{-1}a = e \in H$.
        \item Symmetric: $a \sim b \Leftrightarrow a^{-1}b \in H \Leftrightarrow (a^{-1}b)^{-1} = b^{-1}a \in H \Leftrightarrow b \sim a$.
        \item Transitive: $a \sim b$ and $b \sim c \Leftrightarrow a^{-1}b \in H$ and $b^{-1}c \in H \Leftrightarrow a^{-1}b \cdot b^{-1}c = a^{-1}c \in H \Leftrightarrow a \sim c$.
    \end{enumerate}
    If you take an element and add other elements based on this equivalence relation, you get a subgroup. This is called a \textit{coset}.
\end{definition}





