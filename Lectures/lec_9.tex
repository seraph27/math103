\lecture{9}{}{}
\subsection{Cyclic Subgroups}

\begin{exercise}
    Let $\mathbb{Z}_{12} = \{0, 1, 2, \cdots, 11\}$ and $H$ is the trivial subgroup. What is the smallest subgroup of $\mathbb{Z}_{12}$ that contains $3$?
\end{exercise}
\begin{answer}
    Let H = $\{0, 3, 6, 9\}$, we can see that this is the smallest because we use 3 to generate the other numbers.
    Additionally, $H$ is isomorphic to $\mathbb{Z}_4$.
\end{answer}

\begin{remark}
    If G is a group and H is a subgroup of G.\\
    If $a \in H$ then $a^n \in H \quad \forall \quad n \in \mathbb{Z}$.
    where $a^0 = e$ is the identity element.
\end{remark}

\begin{theorem}
    Let $G$ be a group and $a \in G$ and set H = $\{a^n : n \in \mathbb{Z}\}$, then $H$ is a subgroup, and it's the smallest subgroup 
    of $G$ that contains $a$.
\end{theorem}
\begin{proof}
    \vphantom{}\\
    \begin{enumerate}
        \item $H$ is closed:
        Given $a^r, a^s \in H$, then $(a^r)(a^s) = a^{r+s} \in H$.
        \item $H$ has an identity element: $e = a^0 \in H$.
        \item $H$ has an inverse element: $a^r \in H$, take $a^{-r} \in H$ such that $a^r(a^{-r}) = a^{-r}(a^r) = e$.
    \end{enumerate}
\end{proof}

\begin{definition}
    Let $G$ be a group and $a \in G$. If $H = \{a^n : n \in \mathbb{Z}\}$, then $H$ is called the cyclic subgroup generated by $a$.
    We denote $H = \langle a \rangle$.
\end{definition}

\begin{definition}
    A group G is cyclic if $G = \langle a \rangle$ for some $a \in G$.
\end{definition}

\begin{eg}
    $\mathbb{Z}_n = \{0, 1, 2, \cdots, n-1\}$ is a cyclic group $= \langle 1 \rangle$.
\end{eg}

\begin{eg}
    $\mathbb{Z}_4 = \{0, 1, 2, 3\}$ is a cyclic group $= \langle 1 \rangle$. 1 is a generator for $\mathbb{Z}_4$.
    We can see 3 is also a generator for $\mathbb{Z}_4$. But 2 is not a generator for $\mathbb{Z}_4$.
\end{eg}

\begin{eg}
    $U_n = $ the $n^{th}$ roots of unity.\\
    \[U_n = \{e^{2\pi i k/n} : k = 0, 1, 2, \cdots, n-1\}\]
    So this is a cyclic group generated by $e^{2\pi i/n}$. So $U_n = \langle e^{2\pi i/n} \rangle$.
\end{eg}

\begin{exercise}
    $S_{10}$ is a permutation on $A = \{1, 2, \cdots, 10\}$.
    $\sigma = (1, 10)(2, 9)(3, 8)(4, 7)(5, 6)$\\
    Compute $|\langle \sigma \rangle|$.
\end{exercise}
\begin{answer}
    $\sigma \circ \sigma = (1, 10)(2, 9)(3, 8)(4, 7)(5, 6) \circ (1, 10)(2, 9)(3, 8)(4, 7)(5, 6) = (1)(2)(3)\cdots(10) = i$\\
    So $|\langle \sigma \rangle| = 2$. 
\end{answer}

\section{Cyclic Groups}
\begin{theorem}
    Every cyclic group is abelian.
\end{theorem}
\begin{proof}
    Let $G = \langle a \rangle$ be a cyclic group.\\
    Let $a^r, a^s \in G$.\\
    Then 
    \[(a^r)(a^s) = a^{r+s} = a^{s+r} = (a^s)(a^r)\]
    So $G$ is abelian.
\end{proof}

\begin{eg}
    Let $\mathbb{Z}_{10} = \{0, 1, 2, \cdots, 9\}$.\\
    \[\langle 2 \rangle = \{0, 2, 4, 6, 8\}\] did not generate all of $\mathbb{Z}_{10}$.\\
    \[\langle 3 \rangle = \{0, 3, 6, 9, 2, 5, 8, 1, 4, 7\}\] generated all of $\mathbb{Z}_{10}$.\\
    You can check if they have a common divisor or not to determine if they generate all of $\mathbb{Z}_{10}$.
\end{eg}

\begin{theorem}[Division Algorithm]
    $n=qm+r$
\end{theorem}

\begin{theorem}
    Let $G$ be a cyclic group.
    Then any subgroup of G is also cyclic.
\end{theorem}

