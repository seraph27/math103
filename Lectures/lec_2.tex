\chapter{Group and Subgroups}
\lecture{2}{}{}
\section{Binary Operators}

\begin{definition}
    A binary operation $*$ on $S$ is a function mapping every element in $S \times S$ into $S$
\end{definition}

\begin{exercise}
    Let $M(\mathbb{R})$ = set of all square matrices in $\mathbb{R}$, is $+$ a binary operator on $M$ ?
\end{exercise}
\begin{answer}
    No, because different sized matrices cannot add together.
\end{answer}

\begin{exercise}
    Let $\mathbb{Z}^+ = \{1, 2, 3, ... \} $, then we define $a*b=c$ s.t. c is at least 5 more than $a+b$, is $*$ a binary operator ?
\end{exercise}
\begin{answer}
    No, because the output isn't unique. $1*2 = \{8, 9, 10...\}$.
\end{answer}

\begin{definition}
    If $(S, *)$ is a binary algebraic structure, then $H\subseteq S$ is closed under this operation iff $\forall$ $a, b \in H$, $a*b \in H$
\end{definition}

\begin{note}
    If $M_{2}(\mathbb{R})$ are all $2\times2$ matrices over $\mathbb{R}$, then $(M_{2}(\mathbb{R})$, $+)$ is a proper algebraic structure.\\ 
\end{note}

\begin{exercise}
    If $H\subseteq M_{2}(\mathbb{R})$, $H = \left\{ \begin{bmatrix} a & -b \\ b & a \end{bmatrix} : a,b \in \mathbb{R} \right\}$, is H closed under $+$ ?
\end{exercise}
\begin{answer}
    Yes
\end{answer}
\begin{proof}[Proof]
    $\begin{bmatrix} a & -b \\ b & a \end{bmatrix}$ + 
    $\begin{bmatrix} c & -d \\ d & c \end{bmatrix}$ = 
    $\begin{bmatrix} \color{red}a+c & \color{blue}-(b+d) \\ \color{blue}b+d & \color{red}a+c \end{bmatrix}$ $\in$ H
\end{proof}

\begin{exercise}
    Let $\mathbb{C} = \{a+b\mathit{i} : a, b \subseteq \mathbb{R} \}$, is $\mathbb{C}$ closed under addition and multiplication?
\end{exercise}
\begin{answer}
    Yes, using Euler's formula we know that $a+b\mathit{i} = \sqrt{a^2+b^2} e^{i\theta}$, so it will stay complex under + and $\times$.
\end{answer}

\begin{exercise}
    Let $H \subseteq \mathbb{C}$ and $H = \{a+b\mathit{i} : \sqrt{a^2+b^2} = 1\}$, is H closed under additon / multiplication?
\end{exercise}
\begin{answer}
    It is closed under multiplication but not addition.
\end{answer}

\begin{eg}
    Let $(S, *)$ and $(S^\prime, *)$ be two algebraic structures, we want to show whether they are the same. 
    \begin{answer}
        Need to consider basic properties: $*$ is commutative $\iff$ $a*b=b*a$\\
        Let $\mathcal{F}$ = the set of functions $f: \mathbb{R} \rightarrow \mathbb{R}$, we argue that $f \circ g$  is not commutative
    \end{answer}
    \begin{proof}[Proof]
        $\circ$ is not commutative on $\mathcal{F}$ because lets say h = $\sin(x)$, g = $e^x$, then
        \[h \circ g = h(g(x)) = \sin(e^x) \in \mathcal{F}\]
        \[g \circ h = g(h(x)) = e^{\sin(x)} \in \mathcal{F}\]
        but $\sin(e^x) \neq e^{\sin(x)}$, so back to the question, it may or may not be the same depending on what $*$ is.
    \end{proof}
\end{eg}

\begin{definition}
    If we have a structure $(\mathcal{F}, \circ)$, then $\circ$ is associative, i.e. $f\circ(g\circ h) = (f\circ g)\circ h$
\end{definition}
\begin{proof}
    Computing them shows that they are equal
  \[(f\circ(g\circ h))(x) = f((g\circ h)(x)) = f(g(h(x)))\]
  \[((f\circ g)\circ h)(x) = (f\circ g)(h(x)) = f(g(h(x)))\]
\end{proof}

\begin{exercise}
    $\mathbb{Z}^+ = \{1, 2, 3, 4...\}$, ans define $a*b = 2^{a\cdot b}$, is $(\mathbb{Z}^+$, $*)$ 1. commutative, 2. associative ?
\end{exercise}
\begin{answer}\vphantom{}
    \item  1. Yes, $a*b = 2^{a\cdot b} = 2^{b\cdot a} = b*a$
    \item 2. No, $2*(3*4) \neq (2*3)*4$ 
\end{answer}

\begin{exercise}
    Given $(S, *)$ where $*$ is commutative and associative. Given $H \subseteq S$ where H = $\{a \in S : a*a=a\}$, show that H is closed under $*$.
\end{exercise}
\begin{proof}[Proof]
    $a*a=a$ and $b*b=b$, we can show $[a*b] * [a*b] = [a*b]$ because by associativity and commutativity
    \[[a*b] * [a*b] = a*b * a*b = a*a * b*b = a*b\]
\end{proof}



