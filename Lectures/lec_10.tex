\lecture{10}{}{}

\begin{theorem}
    If $G = \langle a \rangle$ 
    \begin{enumerate}
        \item If $|G| = \infty  \Longrightarrow  G \cong (\mathbb{Z}, + )$ 
        \item If $|G| = n \Longrightarrow  G \cong (\mathbb{Z}_n, +_n )$ 
    \end{enumerate}
\end{theorem}
\begin{proof}
    Case 1:\\
    Suppose $|G| = \infty$, For all positive $m \geq 1$, $a^m \neq e$\\
    Goal is show that $G \cong (\mathbb{Z}, + )$\\
    We need to check all elements in G are distinct. For sake of contradiction, suppose there exists i < j such that:  
    \[a^i = a^j \Rightarrow e = a^{j-i}\]
    But j-i is a positive integer. This contradicts the assumption that $a^m \neq e$ for all positive m, so every element in G is distinct.\\
    So we can define:
    \[\phi : G \rightarrow \mathbb{Z}, \phi(a^i) = i\]
    This is a bijection.\\
    Case 2:\\
    There exists positive $m > 0$ such that $a^m = e$.\\
    Again we define $\phi : G \rightarrow \mathbb{Z}_m$ by $\phi(a^i) = i \mod m$\\
\end{proof}

\begin{eg}
    Let $\mathbb{Z}_{12} = \{0, 1, 2, \cdots, 11\}$ equipped with addition modulo 12.\\
    Let $\langle 3 \rangle = $ the subgroup of $\mathbb{Z}_{12}$ generated by 3.\\
    We get $\langle 3 \rangle = \{0, 3, 6, 9\}$\\
    $|\langle 3 \rangle| = 4$\\
    $\langle 8 \rangle = \{0, 8, 4\}$\\
    $|\langle 8 \rangle| = 3$\\
\end{eg}

\begin{remark}
    The size of a subgroup of a finite cyclic group depends on the divisors. 
\end{remark}

\begin{definition}[Greatest Common Divisor]
    Fix integers r and s. gcd(r, s) is the largest positive integer that divides both r and s.\\ 
\end{definition}

\begin{definition}
    Fix r and s. The gcd(r, s) is the generator of the cyclic subgroup of \[H = \{n\cdot r + m \cdot s : n, m \in \mathbb{Z}\} \leq \mathbb{Z}\]
    $H = \langle \gcd(r, s) \rangle$\\
\end{definition}

\begin{corollary}
    Fix r and s. If there exists $m, n \in \mathbb{Z}$ such that $n \cdot r + m \cdot s = 1$, then $\gcd(r, s) = 1$.\\
    So r and s are coprime.\\
\end{corollary}
\begin{proof}
    \vphantom{}\\
    \begin{prev}
        Recall that let $G = \langle a \rangle$. If G is a cyclic group generated by a, then ANY subgroup of G is also cyclic.
    \end{prev}
    \noindent$(\mathbb{Z}, +) = \langle 1 \rangle$\\
    Fix r and s. $H \subseteq \mathbb{Z}$ and $H = \{m \cdot r + n \cdot s : m, n \in \mathbb{Z}\}$\\
    By the above theorem, $(H, +)$ is cyclic because it is a subgroup of a cyclic group.\\
    Now we also show that H is a subgroup:
    \begin{enumerate}
        \item H is closed under addition: $m_1 \cdot r + n_1 \cdot s + m_2 \cdot r + n_2 \cdot s = (m_1 + m_2) \cdot r + (n_1 + n_2) \cdot s$
        \item Identity: $0 \cdot r + 0 \cdot s = 0$
        \item H is closed under inverses: $m \cdot r + n \cdot s \Rightarrow -m \cdot r + -n \cdot s$, and $(mr+ns) + (-mr-ns) = 0$
    \end{enumerate}
\end{proof}

\begin{eg}
    $\mathbb{Z}_4 = \{0, 1, 2, 3\}$\\
    All subgroups of $\mathbb{Z}_4$ are cyclic.\\
    $\langle 0 \rangle = \{0\}$\\\
    $\langle 1 \rangle = \{0, 1, 2, 3\} \cong \mathbb{Z}_4$\\
    $\langle 2 \rangle = \{0, 2\} \cong \mathbb{Z}_2$\\
    $\langle 3 \rangle = \{0, 1, 2, 3\} \cong \mathbb{Z}_4$\\
\end{eg}

\begin{theorem}
    Let $G = \langle a \rangle$ be a cyclic group of order n.\\
    $G = \{e, a, a^2, \cdots, a^{n-1}\}$\\
    \begin{enumerate}
        \item Let $a^s \in G$, then $|H| = |\langle a^s \rangle| = \frac{n}{\gcd(n, s)}$\\
        \item Moreover, $a^s, a^t \in G$, if $gcd(s, n) = d = gcd(t, n)$, then $\langle a^s \rangle = \langle a^t \rangle$\\
    \end{enumerate}
\end{theorem}
\begin{proof}
    Let m be the smalllest positive integer such that $(a^s)^m = e$.\\ We want to show that $|H| = m = \frac{n}{d}$.\\
    If $(a^s)^m = e$, then $a^{sm} = e = (a^{s \cdot m})$\\ Which will have some multiple of n on the exponent.\\
    Let d = gcd(s, n).\\
    We know $d = u \cdot n + v \cdot s$ for some integers u, v $\in \mathbb{Z}$.\\
    \[1 = u (\frac{n}{d}) + v (\frac{s}{d})\]
    $(\frac{n}{d})$ and $(\frac{s}{d})$ are coprime from the corollary above.\\
    We know $s \cdot m$ is a multiple of n. It follows that $(\frac{sm}{n}) = (\frac{m \frac{s}{d}}{\frac{n}{d}})$ is an integer.\\
    Hence, $(\frac{n}{d})$ must divide m. 
\end{proof}


