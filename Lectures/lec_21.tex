\lecture{21}{}{}

\begin{eg}
    The factor group $G/H$ has left cosets equal to right cosets. We can show that $(G/N, \ast)$ is a group.\\
    \begin{enumerate}
        \item Identity: $eH$ is the identity element.
        \item Inverse: $(aH)^{-1} = a^{-1}H$.
        \item It is closed under $\ast$
        \item Associative: $(aH \ast bH) \ast cH = aH \ast (bH \ast cH)$.
    \end{enumerate}
\end{eg}

\begin{eg}
    Let $G = \mathbb{Z}_3 \times \mathbb{Z}_6$ and $|G| = 18$.\\
    Let $H = \langle (1, 1) \rangle = \{(0, 0), (1, 1), (2, 2), (0, 3), (1, 4), (2, 5)\}$.\\
    $G/H = \{H, (1, 0) + H, (2, 0) + H\} \cong \mathbb{Z}_3$.\\
    What is the order of $(1, 4) + H$?\\
    What is the order of $(2, 1) + H$?\\
\end{eg}
\begin{answer}
    We need to find the minimum n such that $[(1, 4) + H]^n = H$.\\
    $(1, 4) + H = H$, which is the identity in $G/H$, so the order is 1.\\
    $(2, 1) + H$ has order 3.\\
\end{answer}

\begin{theorem}
    The following 4 equivalent conditions are required for subgroup $H$ in $G$ to be normal
    \begin{enumerate}
        \item $ghg^{-1} \in H$ for all $g \in G$ and $h \in H$. 
        \item $gHg^{-1} = \{ghg^{-1} : h \in H\} = H$ for all $g \in G$.
        \item $\exists \phi: G \to G^{\prime}$ such that $H = \ker \phi$.
        \item $gH = Hg$ for all $g \in G$ (normal group).
    \end{enumerate}
\end{theorem}

\begin{eg}
    Let $G$ be a finite group, and let $H$ be a subgroup of $G$.\\
    $H$ is the only subgroup of $G$ with $|H| = d$. Prove that $H$ is normal in $G$.\\
\end{eg}
\begin{proof}
    By the above theorem, it suffices to show that $gHg^{-1} = H$ for all $g \in G$.\\
    For sake of contradiction, $\exists g \in G$ such that $gHg^{-1} \neq H$.\\
    If we can show that $gHg^{-1}$ is a subgroup of $G$, and $|gHg^{-1}| = |H|$, then we have a contradiction.\\ 
    By the automorphism $\phi: g to gHg^{-1}$, we know $|gHg^{-1}| = |H|$ because $\phi$ is bijective.\\
    Now we need to show that $gHg^{-1}$ is a subgroup of $G$.\\
    \begin{enumerate}
        \item Closure: $g(xy)g^{-1} = (gxg^{-1})(gyg^{-1})$.
        \item Identity: $e \in H$, so $geg^{-1} = e \in gHg^{-1}$.
        \item Inverse: $g(x^{-1})g^{-1} = (gxg^{-1})^{-1}$.
    \end{enumerate}
\end{proof}

\begin{definition}[Automorphism]
    An automorphism $\phi: G \to G$ is an isomorphism from $G$ to $G$.\\
    We can define the mapping as $g(x) = gxg^{-1}$.
    \begin{enumerate}
        \item Bijective: Suppose $g(x) = g(y)$, then $gxg^{-1} = gyg^{-1}$, so $x = y$.
        \item Onto: For all $y \in G$, $\exists x \in G$ such that $g(x) = y$. Choose $y = g^{-1}yg$, then $g(g^{-1}yg) = y$.
    \end{enumerate}
\end{definition}
