\lecture{16}{}{}
\section{Finitely Generated Abelian Groups}

\begin{note}
    The motivation for this section is to use known examples of abelian and non-abelian groups and construct larger groups with them via cartisian product.
\end{note}

\begin{theorem}
    Suppose we have n groups $G_1, G_2, ..., G_n$. Then we calculate cartesian product $G = G_1 \times G_2 \times ... \times G_n$ s.t. $(a_1, a_2, ..., a_n) \in G$.
    Define $\ast$ on G where $(a_1, a_2, ..., a_n),(b_1, b_2, ..., b_n) \in G$\\ 
    Then: \[(a_1, a_2, ..., a_n) \ast (b_1, b_2, ..., b_n) = (a_1 b_1, a_2 b_2, ..., a_n b_n)\]
    G is a group with identity $(e_1, e_2, ..., e_n)$ and inverse $(a_1, a_2, ..., a_n)^{-1} = (a_1^{-1}, a_2^{-1}, ..., a_n^{-1})$.
\end{theorem}

\begin{prev}
    Recall two definitions of order:
    \begin{enumerate}
        \item Order of a group: $G = |G|$.
        \item Order of an element: smallest positive integer n s.t. $a^n = e$.
        Moreover, $n = |\langle a \rangle|$.
    \end{enumerate}
\end{prev}

\begin{eg}
$\mathbb{Z}_2 = \{0, 1\}$, and $\mathbb{Z}_3 = \{0, 1, 2\}$.\\
$\mathbb{Z}_2 \times \mathbb{Z}_3 = \{(0, 0), (0, 1), (0, 2), (1, 0), (1, 1), (1, 2)\}$.\\
$| \mathbb{Z}_2 \times \mathbb{Z}_3 | = 2 \times 3 = 6$.\\
\end{eg}

\begin{remark}
    $\mathbb{Z}_2 \times \mathbb{Z}_3 \cong \mathbb{Z}_6$ and is cyclic.
    It's generator is $(1, 1)$.
\end{remark}

\begin{exercise}
    We have $|\mathbb{Z}_3 \times \mathbb{Z}_3| = 9$. Is $\mathbb{Z}_3 \times \mathbb{Z}_3$ cyclic?
\end{exercise}
\begin{answer}
    Find whether there exists an element of order 9.\\
    The answer is no. Suppose $(a, b) \in \mathbb{Z}_3 \times \mathbb{Z}_3$. Then $(a, b) + (a, b) + (a, b) = (3a, 3b) = (0, 0)$.\\
    Therefore the maximum order is 3.
\end{answer}

\begin{theorem}
    The group $\mathbb{Z}_m \times \mathbb{Z}_n$ is cyclic if and only if $\gcd(m, n) = 1$.
\end{theorem}

\begin{corollary}
    $G = \mathbb{Z}_{m1} \times \mathbb{Z}_{m2} \times ... \times \mathbb{Z}_{mn}$ is cyclic if and only if $\gcd(m_1, m_2, ..., m_n) = 1$.
\end{corollary}

\begin{theorem}
    $(a_1, a_2, ..., a_n) \in G = \mathbb{Z}_{m1} \times \mathbb{Z}_{m2} \times ... \times \mathbb{Z}_{mn}$.\\
    If $r_i$ is the order of $a_i$, then $| (a_1, a_2, ..., a_n) | = lcm(r_1, r_2, ..., r_n)$.
\end{theorem}

\begin{exercise}
    $G = \mathbb{Z}_4 \times \mathbb{Z}_{12} \times \mathbb{Z}_{20} \times \mathbb{Z}_{24}$.\\
    \begin{enumerate}
        \item Is G cyclic?
        \item Find the order of G.
        \item $(3, 6, 12, 14) \in G$. Find the order of $(3, 6, 12, 14)$.
    \end{enumerate}
\end{exercise}
\begin{answer}
    \begin{enumerate}
        \item $\gcd(4, 12, 20, 24) = 4 \neq 1$. Therefore G is not cyclic.
        \item $|G| = 4 \times 12 \times 20 \times 24 = 11520$.
        \item  $| (3, 6, 12, 14) | = lcm(4, 2, 5, 3) = 60$.
    \end{enumerate}
\end{answer}


