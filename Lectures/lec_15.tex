\lecture{15}{}{}

\begin{note}
    So far, all groups of permutations we've seen are equipped with the composition operation. 
\end{note}

\begin{eg}
   $ \mathbb{Z}_n$ is not a permtuation group. $\mathbb{Z}_n \cong$ (group of permutations).
\end{eg}

\begin{eg}
 $\sigma^i$ can be defined in two row notation as: 
    \[
        \begin{pmatrix}
            1 & 2 & 3 & 4 & ... & n \\
            1+i & 2+i & 3+i & 4+i & ... & n+i
        \end{pmatrix}
    \]

    $\sigma^n = \sigma^0 = i$.\\

    Also $\langle \sigma \rangle = \{e, \sigma, \sigma^2, ..., \sigma^{n-1}\}$.\\
\end{eg}
\begin{remark}
$\langle \sigma \rangle \cong (\mathbb{Z}_n, +_n)$.
\end{remark}

\begin{exercise}
Let GL(n, R) be the set of all invertible n x n matrices with real entries. Let G = (GL(n, R), $\times$).\\
Is this a permutation group?\\
\end{exercise}
\begin{answer}
Yes, because $A: \mathbb{R}^n \to \mathbb{R}^n$ is a bijection of $\mathbb{R}^n$ if and only if A is invertible.\\
\end{answer}

\begin{theorem}[Cayley's Theorem]
    Every group G is isomorphic to a group of permutations.
\end{theorem}

\begin{corollary}
    Every finite group G is isomorphic to a subgroup of $S_n$ for a sufficiently large n.
\end{corollary}

\begin{definition}[Properties of $S_n$]
   Let $S_n$ be the permutation group on $\{1, 2, ..., n\}$.\\
   $|S_n| = n!$.\\
   Let us define $A_n$ and $B_n$ as follows:
   $A_n$ is the alternating group on $\{1, 2, ..., n\}$, i.e. the set of all even permutations.\\ 
   $B_n$ is the set of all odd permutations.\\
\end{definition}

\begin{definition}
    A cycle of length 2 is called a transposition.\\
\end{definition}

\begin{theorem}
    Any $\sigma \in S_n$ can be written as a product of transpositions.\\
    Another way to think of this is any permutation can be obtained by swapping pairs
\end{theorem}

\begin{exercise}
    Let $\sigma = \begin{pmatrix}
        1 & 2 & 3 & 4 & 5 & 6 & 7 & 8 \\
        5 & 3 & 2 & 8 & 4 & 7 & 6 & 1
    \end{pmatrix}$.\\

    We get $\sigma = (1, 5, 4, 8)(2, 3)(6, 7)$.\\
    $ = (1, 8)(1, 4)(1, 5)(2, 3)(6, 7)$.\\
\end{exercise}

\begin{theorem}
If $\sigma \in S_n$, then $sigma$ cannot be expressed as both an even and an odd number of transpositions.\\
\end{theorem}

\begin{definition}
    $S_n = A_n \cup B_n$.\\
    Where $A_n$ is the set of all even permutations and $B_n$ is the set of all odd permutations.\\
    $|A_n| = |B_n| = \frac{n!}{2}$.\\
\end{definition}