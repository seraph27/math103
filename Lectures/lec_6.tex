\lecture{6}{}{}
\subsection{Circle Algrbra}
\begin{eg}
    Define $\mathbb{C} = \{a + bi \mid a, b \in \mathbb{R}\}$. Then $(\mathbb{C}, +)$ is an abelian group.
\end{eg}
\begin{remark}
    $(\mathbb{C}, \times)$ is not abelian group because 0 does not have an inverse. 
    \begin{note}
        So we come up with a notation $\mathbb{C}^{*} = \mathbb{C} - \{0\}$.
        ($\mathbb{C}^{*}$, $\times$) is an abelian group.
    \end{note}
\end{remark}

\begin{note}[Euler's Formula]
    z $\in$ $\mathbb{C}^{*}$, $z = a + bi$. Then $z = |z|e^{i \theta}$.
    where $|z| = \sqrt{a^2 + b^2}$ and $\theta = \arctan(\frac{b}{a})$.
\end{note}

\begin{eg}
    \begin{enumerate}
        \item Let $u = \{z \in \mathbb{C}^{*}, |z| = 1\}$. Then $(u, \times)$ is an abelian group.
    \end{enumerate}
\end{eg}

\begin{eg}[Roots of Unity]
    Let $n \in \mathbb{N}$. Then $u_n = \{z \in \mathbb{C}^{*}, z^n = 1\}$.
    \begin{enumerate}
        \item $u_1 = \{1\}$.
        \item $u_2 = \{1, -1\}$.
        \item $u_3 = \{1, e^{\frac{2\pi i}{3}}, e^{\frac{4\pi i}{3}}\}$.
        \item $u_4 = \{1, i, -1, -i\}$.
        \item $u_n$ = $\{e^{\frac{2\pi i k}{n}} \mid k = 0, 1, 2, \ldots, n-1\}$.
    \end{enumerate}

    \begin{note}
        $(u_n, \times)$ is an abelian group of order n.
        Also, $u_n \cong \mathbb{Z}_n$.
    \end{note}  
\end{eg}

\section{Non Abelian Groups}
\subsection{Permutation Groups}
\begin{note}[Notation]
    From now on, if $(G, *)$ is a group, we will write a*b as $ab$.\\
    $a^k$ means $a*a*\ldots*a$ (k times).\\
    $a^{-k}$ means $a^{-1} * a^{-1} * \ldots * a^{-1}$ (k times).\\
    Operator should be clear from context so most of the time we will omit it.
\end{note}

\begin{definition}
    The order of a group $G$ is the number of elements in $G$.
\end{definition}

\begin{definition}
    Let A be a set. A permutation of A is a bijection $\phi: A \to A$.
\end{definition}
\begin{eg}
    Let $A = {1, 2, 3, 4, 5}$\\
    Let $\sigma$ be a permutation of A. Then $\sigma = \begin{pmatrix} 1 & 2 & 3 & 4 & 5 \\ 3 & 1 & 5 & 2 & 4 \end{pmatrix}$.
\end{eg}

\begin{definition}
    Let's define a composite operator on $S_A$.
    Let $\sigma, \tau \in S_A$. Then $\sigma \circ \tau$ is a permutation of A defined by $(\sigma \circ \tau)(x) = \sigma(\tau(x))$.
\end{definition}

\begin{theorem}
    A set $(S_A, \circ)$ is a group.
\end{theorem}
\begin{proof}
    \vphantom{}\\
    \begin{enumerate}
        \item Associativity: Let $\sigma, \tau, \rho \in S_A$. Then $(\sigma \circ \tau) \circ \rho = \sigma \circ (\tau \circ \rho)$.
        \item Identity: The identity element is the identity permutation $id(x) = x$.
        \item Inverse: Let $\sigma \in S_A$. Then $\sigma^{-1}$ is the inverse of $\sigma$. This reverse the mapping of $\sigma$.
    \end{enumerate}
\end{proof}