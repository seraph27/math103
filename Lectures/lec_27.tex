\lecture{27}{}{}

\begin{note}
There are 3 types of normal subgroups:
\begin{enumerate}
    \item Maximal normal subgroups (Simple Groups)
    \item Center of G, $Z(G)$
    \item Intersection of 2 normal subgroups, NM
\end{enumerate}
\end{note}

\begin{eg}
Let $G$ be a group, and $H = G$, then $H$ is normal because $ghg^{-1} \in H = G \to$ H is normal.\\
 $|G/H| = 1$ and $G/H \cong \mathbb{Z}_1$.
\end{eg}

\begin{remark}
If $G$ is a group and the only normal subgroups are $H = \{e\}$ and $H = G$, then $G$ does not produce any interesting factor groups.
\end{remark}

\begin{theorem}
Let $G$ be a group. Then G is simple if it has no proper and nontrivial normal subgroups.\\
Trivial means $H = \{e\}$ and Non-proper means $H = G$.
\end{theorem}

\begin{eg}
    $G = \mathbb{Z}_p$ where $p$ is prime. Then $G$ is simple because the only normal subgroups are $H = \{e\}$ and $H = G$.
\end{eg}

\begin{eg}
    $G$ is a group and $|G| = p$ where $p$ is prime. Then $G$ is also simple for the same reason above.
\end{eg}

\begin{theorem}
If $n \geq 5$, then $A_n$ is simple.
\end{theorem}

\begin{eg}
Let $G$ be a group and there exists a subgroup $H \subseteq G$ such that $[G:H] = 2$. Is G simple?\\
\end{eg}
\begin{answer}
    This means the number of left cosets is 2, so $aH = Ha$ so $H$ is normal. $|H| \neq 1$, thus $G$ is not simple.
\end{answer}

\begin{definition}
    $H \subseteq G$ is a maximal normal subgroup if:
    \begin{enumerate}
        \item $H$ is normal.
        \item There exists no proper nontrivial normal subgroup that contains $H$.
    \end{enumerate}
\end{definition}

\begin{remark}
It is possible that $H$ is a maximal subgroup, but there are larger subgroups that do not contain $H$.
\end{remark}

\begin{theorem}
    Let $G$ be a group and $H$ be a subgroup. $H$ is maximal normal subgroup if and only if $G/H$ is simple.
\end{theorem}

\begin{definition}
    The center of a group $G$, denoted by $Z(G) = \{g \in G \mid gh = hg \ \forall h \in G\}$.
\end{definition}

\begin{theorem}
Let $G$ be a group. Then $Z(G)$ is a normal subgroup of $G$.
\end{theorem}
\begin{proof}
    First, check for subgroup properties:
    \begin{enumerate}
        \item Closed: Let $Z_1, Z_2 \in Z(G)$ then for all $g \in G$, $g(Z_1Z_2) = Z_1gZ_2 = (Z_1Z_2)g$
        \item Identity: $e \in Z(G)$ because $eg = ge$ for all $g \in G$.
        \item Inverse: Let $z \in Z(G)$, then $z^{-1} \in Z(G)$ because for all $g \in G$, \[gzz^{-1} = g = zgz^{-1} \implies g(z^{-1}) = (z^{-1})g\]
    \end{enumerate}
    Now check for normality:
    \[gzg^{-1} = z \ \forall z \in Z(G) \implies gZ(G)g^{-1} = Z(G)\]
\end{proof}
