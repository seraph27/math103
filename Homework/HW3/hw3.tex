\documentclass[12pt]{article}
\usepackage{amsmath, amsfonts, mathtools, amsthm, amssymb}
\usepackage{graphicx}			% Use this package to include images
\usepackage{amsmath}			% A library of many standard math expressions
\usepackage[margin=1in]{geometry}% Sets 1in margins. 
\usepackage{fancyhdr}			% Creates headers and footers
\usepackage{enumerate}          %These two package give custom labels to a list
\usepackage[shortlabels]{enumitem}
\newcommand{\ind}{\hspace{0.2cm}}
\newcommand{\sol}{\setlength{\parindent}{0cm}\textbf{\textit{Ans:}}\setlength{\parindent}{1cm} }
\usepackage{listings}
\usepackage{xcolor}
\lstset{ 
  frame=single,                           % Change to 'single' for a cleaner look
  language=C++,                           % Language of the code
  aboveskip=2mm,                         % Space above the code
  belowskip=2mm,                         % Space below the code
  showstringspaces=false,                % Do not show spaces in strings
  columns=flexible,                      % Column alignment
  basicstyle={\footnotesize\ttfamily},   % Set the basic font size and family
  numbers=left,                          % Show line numbers on the left
  numberstyle=\tiny\color{gray},         % Style for line numbers
  keywordstyle=\color{blue},             % Style for keywords
  commentstyle=\color{red!80!black},     % Darken the comment color
  stringstyle=\color{green!80!black},    % Darken the string color
  backgroundcolor=\color{gray!10},       % Light gray background for the code
  breaklines=true,                       % Enable line breaking
  breakatwhitespace=true,                % Break at whitespace
  tabsize=4,                             % Set tab size
  literate={~} {$\sim$}{1}               % Change the tilde character
}

% Creates the header and footer. You can adjust the look and feel of these here.
\pagestyle{fancy}
\fancyhead[l]{Yi-An Chao}
\fancyhead[c]{Math 103A Homework \#3}
\fancyhead[r]{\today}
\fancyfoot[c]{\thepage}
\renewcommand{\headrulewidth}{0.2pt} %Creates a horizontal line underneath the header


\begin{document} %The writing for your homework should all come after this. 
%Enumerate starts a list of problems so you can put each homework problem after each item. 
\begin{enumerate}[start=1,label={\bfseries Question \arabic*:},leftmargin=1in] %You can change "Problem" to be whatevber label you li
  \item[\textbf{\#4.12}] Compute the permutation products.
  \begin{enumerate}
      \item[(a)] \( (1, 5, 2, 4)(1, 5, 2, 3) \)
      \item[(b)] \( (1, 5, 3)(1, 2, 3, 4, 5, 6)(1, 5, 3)^{-1} \)
      \item[(c)] \( [(1, 6, 7, 2)^2 (4, 5, 2, 6)^{-1} (1, 7, 3)]^{-1} \)
      \item[(d)] \( (1, 6)(1, 5)(1, 4)(1, 3)(1, 2) \)
  \end{enumerate}
  \sol{}
  \begin{enumerate}
    \item[(a)] We start with 3:
    \[\sigma(3) = 5, 
      \sigma(5) = 4, 
      \sigma(4) = 1, 
      \sigma(1) = 2, 
      \sigma(2) = 3
    \]
    Thus, \( (1, 5, 2, 4)(1, 5, 2, 3) = (3, 5, 4, 1, 2) = (1, 2, 3, 5, 4) \)
    \item[(b)] First calculate \( (1, 5, 3)^{-1} = (1, 3, 5) \). Then we can calculate the product of \( (1, 5, 3)(1, 2, 3, 4, 5, 6)(1, 3, 5) \):
    \[\sigma(1) = 4, 
    \sigma(4) = 3, 
    \sigma(3) = 6,
    \sigma(6) = 5, 
    \sigma(5) = 2,
    \sigma(2) = 1 
    \]
    Thus \( (1, 5, 3)(1, 2, 3, 4, 5, 6)(1, 3, 5) = (1, 4, 3, 6, 5, 2) \)
    \item[(c)] First calculate \( [(1, 6, 7, 2)^2 (4, 5, 2, 6)^{-1} (1, 7, 3)]^{-1}\) \\
    \( = (2, 7, 6, 1)(2, 7, 6, 1)(4, 5, 2, 6)(3, 7, 1) \) Then:
    \[\sigma(1) = 3,
    \sigma(3) = 1,
    \sigma(2) = 2,
    \sigma(4) = 5,
    \sigma(5) = 6,
    \sigma(6) = 4,
    \sigma(7) = 7,
    \]
    Thus \( [(1, 6, 7, 2)^2 (4, 5, 2, 6)^{-1} (1, 7, 3)]^{-1} = (1, 3)(2)(4, 5, 6)(7) \)
    \item[(d)] \( (1, 6)(1, 5)(1, 4)(1, 3)(1, 2) \)
    \[
    \sigma(1) = 2,
    \sigma(2) = 3,
    \sigma(3) = 4,
    \sigma(4) = 5,
    \sigma(5) = 6,
    \sigma(6) = 1
    \]
    Thus \( (1, 6)(1, 5)(1, 4)(1, 3)(1, 2) = (1, 2, 3, 4, 5, 6) \)
  \end{enumerate}
  \item[\textbf{\#4.26}] \( f_3 : \mathbb{R} \to \mathbb{R} \) defined by \( f_3(x) = -x^3 \). Is $f_3$ a permutation?\\
  \sol{}\\
  Recall that a permutation is one-to-one and onto function. Thus, we check these two properties.
  \begin{enumerate}
    \item \( f_3 \) is one-to-one: \\
    Assume \( f_3(x) = f_3(y) \), then \( -x^3 = -y^3 \) implies \( x = y \). Thus, \( f_3 \) is one-to-one.
    \item \( f_3 \) is onto: \\
    For any \( y \in \mathbb{R} \), we can find \( x = -\sqrt[3]{y} \in \mathbb{R} \) such that \\
    \( f_3(x) = -x^3 = -(-\sqrt[3]{y})^3 = y \). Thus, \( f_3 \) is onto.
  \end{enumerate}
  So since \( f_3 \) is one-to-one and onto, it is a permutation.
  \item[\textbf{\#4.35}] Give a careful proof using the definition of isomorphism that if G and $G'$ are both groups with G abelian and
  $G'$ not abelian, then G and $G'$ are not isomorphic.\\
  \sol{}\\
  Recall that a group isomorphism needs to satisfy the following:
  \begin{enumerate}
    \item $\phi$ is a homomorphism, i.e., $\forall a, b \in G, \phi(a * b) = \phi(a) * \phi(b)$.
    \item $\phi$ is bijective, i.e., $\phi$ is both injective (one-to-one) and surjective (onto).
  \end{enumerate} 
  Since $G$ is abelian, we can apply the homomorphism $\phi$, we know \[\phi(a*b) = \phi(b*a) = \phi(a)*\phi(b) = \phi(b)*\phi(a)\]
  Since $\phi$ is an isomorphism and $\phi(a)*\phi(b) = \phi(b)*\phi(a)$, we know $G'$ is abelian. 
  However we already know $G'$ is not abelian so we have a contradiction.\\
   So, $G$ and $G'$ are not isomorphic.

  \item[\textbf{\#5.08}] Is the n $\times$ n matrices with determinant greater than or equal than 1 a subgroup of 
  $GL(n, \mathbb{R})$?\\
  \sol{}\\
  We check the properties of a subgroup:
  \begin{enumerate}
    \item Closure: \\
    Let \( A, B \) be two \( n \times n \) matrices with det $\geq$ 1. 
    Then \( \det(AB) = \det(A)\det(B) \geq 1 \).
    So it is closed under $GL(n, \mathbb{R})$.
    \item Identity: \\
    The identity matrix has determinent 1, which is in $GL(n, \mathbb{R})$.
    \item Inverse: \\
    Let \( A \) be an \( n \times n \) matrix with det $\geq 1$.
    Then \( \det(A^{-1}) = \frac{1}{\det(A)} \leq 1 \).
    So it fails the inverse property so it is NOT a subgroup.
  \end{enumerate}
  
  \item[\textbf{\#5.22}] Describe all the elements in the cyclic subgroup of $GL(2, \mathbb{R})$ generated by the given 2 $\times$ 2 matrix.
  \[
  \begin{bmatrix}
  0 & -1 \\
  -1 & 0
  \end{bmatrix}
  \]
  \sol{}\\
  We try multiply the matrix by itself:
  \[
  \begin{bmatrix}
  0 & -1 \\
  -1 & 0
  \end{bmatrix}
  \begin{bmatrix}
  0 & -1 \\
  -1 & 0
  \end{bmatrix}
  =
  \begin{bmatrix}
  1 & 0 \\
  0 & 1
  \end{bmatrix}
  \]
  Which is the identity matrix, so the cyclic subgroup generated is:
  \[
  \left\{
  \begin{bmatrix}
  0 & -1 \\
  -1 & 0
  \end{bmatrix},
  \begin{bmatrix}
  1 & 0 \\
  0 & 1
  \end{bmatrix}
  \right\}
  \]
  \item[\textbf{\#5.30}]
  Find the order of the cyclic subgroup of $\mathbb{Z}_{10}$ generated by 8\\
  \sol{}\\
  We can calculate the order of the cyclic subgroup generated by 8:\\
  \[
  \begin{array}{c|c}
  n & 8^n \mod 10 \\
  \hline
  1 & 8 \\
  2 & 4 \\
  3 & 2 \\
  4 & 6 \\
  5 & 8 \\
  \end{array}
  \]
  We repeated at n=5 so the order of by 8 is 4.
  \item[\textbf{\#5.56}] Show that if $a \in G$, where $G$ is a finite group with identity $e$, 
  then there exists $n \in \mathbb{Z}^+$ such that $a^n = e$.\\
  \sol{}\\
  We can use the idea of the pigeonhole principle. Since $G$ is finite, there is only a finite number of elements in $G$.
  So there must be two powers of $a$ that are equal, for exmaple, $a^i = a^j$ for some $i > j$.
  Then we can multiply both sides by $a^{-i}$ to get $a^{j-i} = e$. So there exists $n = j-i$ such that $a^n = e$.
  \item[\textbf{\#5.61}] For sets $H$ and $K$, we define the \textit{intersection} $H \cap K$ by
  \[
  H \cap K = \{x \mid x \in H \text{ and } x \in K\}.
  \]
  
  
  \noindent Show that if $H \leq G$ and $K \leq G$, then $H \cap K \leq G$. 
  (Remember: $\leq$ denotes ``is a subgroup of," not ``is a subset of.")\\
  \sol{}\\
  We need to show that $H \cap K$ is a subgroup of $G$. We can check the properties of a subgroup:
  \begin{enumerate}
    \item Closure: \\
    Let \( a, b \in H \cap K \), then by definition of intersection \( a, b \in H \) and \( a, b \in K \). 
    Since \( H \) and \( K \) are subgroups of \( G \), \( ab \in H \) and \( ab \in K \).\\
    Thus, \( ab \in H \cap K \).
    \item Identity: \\
    Since \( H \) and \( K \) are subgroups of \( G \), they both have \( e \) in them.\\
    Thus, \( e \in H \cap K \).
    \item Inverse: \\
    Let \( a \in H \cap K \), then \( a \in H \) and \( a \in K \).
    Since \( H \) and \( K \) are subgroups of \( G \), \( a^{-1} \in H \) and \( a^{-1} \in K \).\\
    Thus, \( a^{-1} \in H \cap K \).\\
    So all three properties are satisfied, and \( H \cap K \) is a subgroup of \( G \).
  \end{enumerate}

\end{enumerate}

\end{document}
