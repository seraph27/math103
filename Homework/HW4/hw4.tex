\documentclass[12pt]{article}
\usepackage{amsmath, amsfonts, mathtools, amsthm, amssymb}
\usepackage{graphicx}			% Use this package to include images
\usepackage{amsmath}			% A library of many standard math expressions
\usepackage[margin=1in]{geometry}% Sets 1in margins. 
\usepackage{fancyhdr}			% Creates headers and footers
\usepackage{enumerate}          %These two package give custom labels to a list
\usepackage[shortlabels]{enumitem}
\newcommand{\ind}{\hspace{0.2cm}}
\newcommand{\sol}{\setlength{\parindent}{0cm}\textbf{\textit{Ans:}}\setlength{\parindent}{1cm} }
\usepackage{listings}
\usepackage{xcolor}
\lstset{ 
  frame=single,                           % Change to 'single' for a cleaner look
  language=C++,                           % Language of the code
  aboveskip=2mm,                         % Space above the code
  belowskip=2mm,                         % Space below the code
  showstringspaces=false,                % Do not show spaces in strings
  columns=flexible,                      % Column alignment
  basicstyle={\footnotesize\ttfamily},   % Set the basic font size and family
  numbers=left,                          % Show line numbers on the left
  numberstyle=\tiny\color{gray},         % Style for line numbers
  keywordstyle=\color{blue},             % Style for keywords
  commentstyle=\color{red!80!black},     % Darken the comment color
  stringstyle=\color{green!80!black},    % Darken the string color
  backgroundcolor=\color{gray!10},       % Light gray background for the code
  breaklines=true,                       % Enable line breaking
  breakatwhitespace=true,                % Break at whitespace
  tabsize=4,                             % Set tab size
  literate={~} {$\sim$}{1}               % Change the tilde character
}

% Creates the header and footer. You can adjust the look and feel of these here.
\pagestyle{fancy}
\fancyhead[l]{Yi-An Chao}
\fancyhead[c]{Math 103A Homework \#4}
\fancyhead[r]{\today}
\fancyfoot[c]{\thepage}
\renewcommand{\headrulewidth}{0.2pt} %Creates a horizontal line underneath the header


\begin{document} %The writing for your homework should all come after this. 

\begin{enumerate}

  
  \item given two $n\times n$ matrices $A=(a_{ij})$ and $B=(b_{uv})$. QAP asks for a permutation $\pi$ of $\{1,\dots,n\}$ minimizing
    \[
      \sum_{i=1}^n\sum_{j=1}^n a_{ij}\,b_{\pi(i)\,\pi(j)}.
    \]

  \item
    \begin{itemize}
      \item let $A$ be the adjacency matrix of your graph:
        \[
          a_{ij} = 
          \begin{cases}
            1, & \text{if }(i,j)\in E,\\
            0, & \text{otherwise.}
          \end{cases}
        \]
      \item let $B$ encode label–products:
        \[
          b_{uv} = u \times v.
        \]
    \end{itemize}

  \item 
    assign labels $\ell(0),\dots,\ell(n-1)$ is the same as choosing a permutation $\pi$, and we want
    \[
      \max_{\pi}\sum_{(i,j)\in E} \ell(i)\,\ell(j)
      \;=\;
      \max_{\pi}\sum_{i<j} a_{ij}\,\pi(i)\,\pi(j).
    \]
    \[
      \max_{\pi}\sum_{i<j} a_{ij}\,\pi(i)\,\pi(j)
      \;=\;
      -\,\min_{\pi}\sum_{i<j} a_{ij}\,\pi(i)\,\pi(j),
    \]
    which (up to the usual factor of 2 in symmetric sums) is exactly
    \[
      \min_{\pi}\sum_{i=1}^n\sum_{j=1}^n a_{ij}\;b_{\pi(i)\,\pi(j)}.
    \]
\end{enumerate}

\end{document}
