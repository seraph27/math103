\documentclass[12pt]{article}
\usepackage{amsmath, amsfonts, mathtools, amsthm, amssymb, mathrsfs}
\usepackage{graphicx}			% Use this package to include images
\usepackage{amsmath}			% A library of many standard math expressions
\usepackage[margin=1in]{geometry}% Sets 1in margins. 
\usepackage{fancyhdr}			% Creates headers and footers
\usepackage{enumerate}          %These two package give custom labels to a list
\usepackage[shortlabels]{enumitem}
\newcommand{\ind}{\hspace{0.2cm}}
\newcommand{\sol}{\setlength{\parindent}{0cm}\textbf{\textit{Ans:}}\setlength{\parindent}{1cm} }
\usepackage{listings}
\usepackage{xcolor}
\lstset{ 
  frame=single,                           % Change to 'single' for a cleaner look
  language=C++,                           % Language of the code
  aboveskip=2mm,                         % Space above the code
  belowskip=2mm,                         % Space below the code
  showstringspaces=false,                % Do not show spaces in strings
  columns=flexible,                      % Column alignment
  basicstyle={\footnotesize\ttfamily},   % Set the basic font size and family
  numbers=left,                          % Show line numbers on the left
  numberstyle=\tiny\color{gray},         % Style for line numbers
  keywordstyle=\color{blue},             % Style for keywords
  commentstyle=\color{red!80!black},     % Darken the comment color
  stringstyle=\color{green!80!black},    % Darken the string color
  backgroundcolor=\color{gray!10},       % Light gray background for the code
  breaklines=true,                       % Enable line breaking
  breakatwhitespace=true,                % Break at whitespace
  tabsize=4,                             % Set tab size
  literate={~} {$\sim$}{1}               % Change the tilde character
}

% Creates the header and footer. You can adjust the look and feel of these here.
\pagestyle{fancy}
\fancyhead[l]{Yi-An Chao}
\fancyhead[c]{Math 103A Homework \#1}
\fancyhead[r]{\today}
\fancyfoot[c]{\thepage}
\renewcommand{\headrulewidth}{0.2pt} %Creates a horizontal line underneath the header


\begin{document} %The writing for your homework should all come after this. 
%Enumerate starts a list of problems so you can put each homework problem after each item. 
\begin{enumerate}[start=1,label={\bfseries Question \arabic*:},leftmargin=1in] %You can change "Problem" to be whatevber label you li
  \item[\textbf{\#0.06}] $\{ n \in \mathbb{Z} \, | \, n^2 < 0 \}$\\
  \sol{$\varnothing$}
  
  \item [\textbf{\#0.12}]
  Let \( A = \{ 1, 2, 3 \} \) and \( B = \{ 2, 4, 6 \} \). For each relation between \( A \) and \( B \) given as a subset of \( A \times B \), decide whether it is a function mapping \( A \) into \( B \). If it is a function, decide whether it is one-to-one and whether it is onto \( B \).
  
  \begin{enumerate}[label=\alph*.]
      \item \( \{ (1, 2), (2, 6), (3, 4) \} \) \sol{Yes, a one-to-one and onto function}
      \item \( \{ (1, 3) \} \text{ and } \{ (5, 7) \} \) \sol{Not a function}
      \item \( \{ (1, 6), (1, 2), (1, 4) \} \) \sol{Not a function}
      \item \( \{ (2, 2), (3, 6), (1, 6) \} \) \sol{Yes, onto function}
      \item \( \{ (1, 6), (2, 6), (3, 6) \} \) \sol{Yes, onto functgion}
      \item \( \{ (1, 2), (2, 6) \} \) \sol{Not a function}
  \end{enumerate}

  \item [\textbf{\#0.16}]
  List the elements of the power set of the given set and give the cardinality of the power set:
  \begin{enumerate}[label=\alph*.]
      \item \( \varnothing \) \sol{$\{\varnothing\}$} : cardinality 1
      \item \( \{ a \} \) \sol{$\{\varnothing, a\}$} : cardinality 2
      \item \( \{ a, b \} \) \sol{$\{\varnothing, a, b, ab\}$} : cardinality 4
      \item \( \{ a, b, c \} \) \sol{$\{\varnothing, a, b, c, ab, ac, bc, abc\}$} : cardinality 8
  \end{enumerate}   

  \item[\textbf{\#0.30}] 
  $\bullet$ \ind $x \mathscr{R} y$  in $\mathbb{R}$ if $x \geq y$ \\
  Determine whether the given relation is an equivalence relation on the set. Describe the partition arising from each equivalence relation.\\
  \sol{
    This is NOT an equivalence relation because it does not satisfy symmetric properities. Example: $2 \geq 1$ but $1 \ngeq  2$.
  }

  \item[\textbf{\#1.03}]
  Compute \( (b * d) * c \) and \( b * (d * c) \). Can you say on the basis of this computation whether \( * \) is associative?\\
  \sol{No, because \( (b * d) * c = e*c = a \), but \( b * (d * c) = b*b = c\)}

  \item[\textbf{\#1.10}] Let \( * \) be defined on \( \mathbb{Z}^+ \) by letting \( a * b = 2^{ab} \).\\
  Determine whether the operation $*$ is associative, whether the operation is commutative, and whether the set has an identity element.\\
  \sol
  {It is commutative because $a*b = b*a = 2^{ab}$, but it is not associative because $(a*b)*c = 2^{2^{ab}c}$, but $a*(b*c) = 2^{2^{bc}a}$. There is no identity element for this set.
  
  }
  \pagebreak
  \item[\textbf{\#1.27}] Let \( H \) be the subset of \( M_2(\mathbb{R}) \) consisting of all matrices of the form 
  \[
  \begin{bmatrix} a & -b \\ b & a \end{bmatrix}
  \]
  for \( a, b \in \mathbb{R} \). Is \( H \) closed under:
  \begin{itemize}
      \item[(a)] matrix addition? \\
      \sol{
        Yes, it is closed under addition\\
        $\begin{bmatrix} a & -b \\ b & a \end{bmatrix}$ + 
        $\begin{bmatrix} c & -d \\ d & c \end{bmatrix}$ = 
        $\begin{bmatrix} \color{red}a+c & \color{blue}-(b+d) \\ \color{blue}b+d & \color{red}a+c \end{bmatrix}$
      }
      \item[(b)] matrix multiplication? \\
      \sol{
        Yes, it is closed under multiplication\\
        $\begin{bmatrix} a & -b \\ b & a \end{bmatrix}$ 
        $\begin{bmatrix} c & -d \\ d & c \end{bmatrix}$ = 
        $\begin{bmatrix} \color{red}ac-bd & \color{blue}-(ad+bc) \\ \color{blue}ad+bc & \color{red}ac-bd \end{bmatrix}$
      }
  \end{itemize}

  \item[\textbf{\#2.02}]
  Let \( * \) be defined on \( 2\mathbb{Z} = \{ 2n \mid n \in \mathbb{Z} \} \) by letting \( a * b = a + b \). 
  Determine whether the binary operation $*$ gives a group structure on the given set.\\
  \sol{
    Yes, all three axioms, associativity, indentity element and inverse holds for this set.
  }
\end{enumerate}

\end{document}
